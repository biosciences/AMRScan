\section{Summary}\label{summary}

\textbf{AMRScan} is a hybrid bioinformatics toolkit implemented in both
R and \href{https://www.nextflow.io/}{Nextflow} for the rapid and
reproducible detection of antimicrobial resistance (AMR) genes from
next-generation sequencing (NGS) data. The toolkit enables users to
identify AMR gene hits in sequencing reads by aligning them against
reference databases such as CARD using BLAST.

The R implementation provides a concise, script-based approach suitable
for single-sample analysis, teaching, and rapid prototyping. In
contrast, the Nextflow implementation enables reproducible, scalable
workflows for multi-sample batch processing in high-performance
computing (HPC) and containerized environments. It leverages modular
pipeline design with support for automated database setup, quality
control, conversion, BLAST alignment, and results parsing.

AMRScan helps bridge the gap between lightweight exploratory analysis
and production-ready surveillance pipelines, making it suitable for both
research and public health genomics applications.

\section{Statement of Need}\label{statement-of-need}

While several large-scale AMR detection platforms exist, many are
resource-intensive or require complex installations. AMRScan addresses
the need for a minimal, transparent, and reproducible toolkit that can
be used flexibly in small labs, clinical settings, or large-scale
surveillance workflows.

The inclusion of a pure Nextflow implementation enables high-throughput,
multi-sample analyses in cloud and HPC environments, while the
standalone R script remains accessible to users in data science,
microbiology, and epidemiology. Both versions use shared components
(e.g., a BLAST parsing script) to ensure consistency and reproducibility
of results.

\section{Usage Guidance}\label{usage-guidance}

AMRScan provides two usage modes, tailored to user needs:

\begin{itemize}
\item
  \textbf{R script (\texttt{AMRScan\_standalone.R})}: Best suited for
  small datasets, single-sample analysis, quick local tests, educational
  use, and lightweight environments without workflow managers.
\item
  \textbf{Nextflow workflow (\texttt{main.nf})}: Designed for
  large-scale, automated analyses, this version excels in multi-sample
  settings, HPC/cloud infrastructure, and environments where
  reproducibility, parallelism, and containerization are priorities.
\end{itemize}

This flexible dual-mode implementation ensures that AMRScan can serve
both teaching/demo scenarios and production-grade bioinformatics
pipelines.

\section{Implementation}\label{implementation}

\begin{itemize}
\tightlist
\item
  The R script \texttt{scripts/AMRScan\_standalone.R} encapsulates the
  entire pipeline in a linear script-based format.
\item
  The Nextflow workflow \texttt{workflow/main.nf} organizes the same
  logic into modular processes:

  \begin{itemize}
  \tightlist
  \item
    \texttt{DownloadCARD}, \texttt{MakeBLASTdb}, \texttt{ConvertFASTQ},
    \texttt{RunBLAST}, and \texttt{ParseResults}
  \end{itemize}
\item
  The shared R script \texttt{scripts/parse\_blast.R} is used for
  post-BLAST result summarization.
\item
  Both implementations are documented, testable, and validated using
  mock NGS input.
\end{itemize}

\section{Acknowledgements}\label{acknowledgements}

The author thanks Professor Vitali Sintchenko and the Sydney Infectious
Diseases Institute (SydneyID) for supporting research in pathogen
genomics and antimicrobial resistance surveillance. This project was
inspired by translational needs in microbial diagnostics and outbreak
detection.

\section{Example Dataset and
Demonstration}\label{example-dataset-and-demonstration}

The example data used to validate AMRScan was obtained from a study by
Munim et al.~(2024) on multidrug-resistant \emph{Klebsiella pneumoniae}
isolated from poultry in Noakhali, Bangladesh {[}@munim2024mdr{]}. The
assembled genome was downloaded from GenBank
(\href{https://www.ncbi.nlm.nih.gov/assembly/GCA_037966445.1}{GCA\_037966445.1}).

For antimicrobial resistance gene detection, we used the protein homolog
model from the Comprehensive Antibiotic Resistance Database (CARD),
version Broadstreet v4.0.1, available at:\\
\url{https://card.mcmaster.ca/download/0/broadstreet-v4.0.1.tar.bz2}

A sample output summary is shown below:

\subsection{Top AMR Hits Summary}\label{top-amr-hits-summary}

\begin{longtable}[]{@{}
  >{\raggedright\arraybackslash}p{(\linewidth - 12\tabcolsep) * \real{0.1718}}
  >{\raggedright\arraybackslash}p{(\linewidth - 12\tabcolsep) * \real{0.2515}}
  >{\raggedright\arraybackslash}p{(\linewidth - 12\tabcolsep) * \real{0.0613}}
  >{\raggedright\arraybackslash}p{(\linewidth - 12\tabcolsep) * \real{0.0491}}
  >{\raggedright\arraybackslash}p{(\linewidth - 12\tabcolsep) * \real{0.0491}}
  >{\raggedright\arraybackslash}p{(\linewidth - 12\tabcolsep) * \real{0.0613}}
  >{\raggedright\arraybackslash}p{(\linewidth - 12\tabcolsep) * \real{0.3558}}@{}}
\toprule\noalign{}
\begin{minipage}[b]{\linewidth}\raggedright
Query
\end{minipage} & \begin{minipage}[b]{\linewidth}\raggedright
Subject
\end{minipage} & \begin{minipage}[b]{\linewidth}\raggedright
Identity
\end{minipage} & \begin{minipage}[b]{\linewidth}\raggedright
Length
\end{minipage} & \begin{minipage}[b]{\linewidth}\raggedright
Evalue
\end{minipage} & \begin{minipage}[b]{\linewidth}\raggedright
Bitscore
\end{minipage} & \begin{minipage}[b]{\linewidth}\raggedright
Annotation
\end{minipage} \\
\midrule\noalign{}
\endhead
\bottomrule\noalign{}
\endlastfoot
NZ\_JBBPBW010000028.1 &
gb\textbar BAM10414.1\textbar ARO:3003039\textbar OprA & 40.839 & 453 &
0 & 252 & OprA {[}Pseudomonas aeruginosa{]} \\
NZ\_JBBPBW010000035.1 &
gb\textbar ABV89601.1\textbar ARO:3004826\textbar LAP-2 & 100.000 & 285
& 0 & 587 & LAP-2 {[}Enterobacter cloacae{]} \\
NZ\_JBBPBW010000001.1 &
gb\textbar CAA66729.1\textbar ARO:3001070\textbar SHV-11 & 100.000 & 286
& 0 & 581 & SHV-11 {[}Klebsiella pneumoniae{]} \\
NZ\_JBBPBW010000010.1 &
gb\textbar CCI79240.1\textbar ARO:3005047\textbar eptB & 99.303 & 574 &
0 & 1109 & eptB {[}Klebsiella pneumoniae subsp. rhinoscleromatis{]} \\
NZ\_JBBPBW010000104.1 &
gb\textbar AHK10285.1\textbar ARO:3002859\textbar dfrA14 & 98.726 & 157
& 0 & 327 & dfrA14 {[}Escherichia coli{]} \\
NZ\_JBBPBW010000011.1 &
gb\textbar AAC75271.1\textbar ARO:3003952\textbar YojI & 83.912 & 547 &
0 & 885 & YojI {[}Escherichia coli str. K-12 substr. MG1655{]} \\
\end{longtable}

\section{References}\label{references}
