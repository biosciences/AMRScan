% Options for packages loaded elsewhere
\PassOptionsToPackage{unicode}{hyperref}
\PassOptionsToPackage{hyphens}{url}
\documentclass[
]{article}
\usepackage{xcolor}
\usepackage[margin=1in]{geometry}
\usepackage{amsmath,amssymb}
\setcounter{secnumdepth}{5}
\usepackage{iftex}
\ifPDFTeX
  \usepackage[T1]{fontenc}
  \usepackage[utf8]{inputenc}
  \usepackage{textcomp} % provide euro and other symbols
\else % if luatex or xetex
  \usepackage{unicode-math} % this also loads fontspec
  \defaultfontfeatures{Scale=MatchLowercase}
  \defaultfontfeatures[\rmfamily]{Ligatures=TeX,Scale=1}
\fi
\usepackage{lmodern}
\ifPDFTeX\else
  % xetex/luatex font selection
\fi
% Use upquote if available, for straight quotes in verbatim environments
\IfFileExists{upquote.sty}{\usepackage{upquote}}{}
\IfFileExists{microtype.sty}{% use microtype if available
  \usepackage[]{microtype}
  \UseMicrotypeSet[protrusion]{basicmath} % disable protrusion for tt fonts
}{}
\makeatletter
\@ifundefined{KOMAClassName}{% if non-KOMA class
  \IfFileExists{parskip.sty}{%
    \usepackage{parskip}
  }{% else
    \setlength{\parindent}{0pt}
    \setlength{\parskip}{6pt plus 2pt minus 1pt}}
}{% if KOMA class
  \KOMAoptions{parskip=half}}
\makeatother
\usepackage{longtable,booktabs,array}
\usepackage{calc} % for calculating minipage widths
% Correct order of tables after \paragraph or \subparagraph
\usepackage{etoolbox}
\makeatletter
\patchcmd\longtable{\par}{\if@noskipsec\mbox{}\fi\par}{}{}
\makeatother
% Allow footnotes in longtable head/foot
\IfFileExists{footnotehyper.sty}{\usepackage{footnotehyper}}{\usepackage{footnote}}
\makesavenoteenv{longtable}
\usepackage{graphicx}
\makeatletter
\newsavebox\pandoc@box
\newcommand*\pandocbounded[1]{% scales image to fit in text height/width
  \sbox\pandoc@box{#1}%
  \Gscale@div\@tempa{\textheight}{\dimexpr\ht\pandoc@box+\dp\pandoc@box\relax}%
  \Gscale@div\@tempb{\linewidth}{\wd\pandoc@box}%
  \ifdim\@tempb\p@<\@tempa\p@\let\@tempa\@tempb\fi% select the smaller of both
  \ifdim\@tempa\p@<\p@\scalebox{\@tempa}{\usebox\pandoc@box}%
  \else\usebox{\pandoc@box}%
  \fi%
}
% Set default figure placement to htbp
\def\fps@figure{htbp}
\makeatother
% definitions for citeproc citations
\NewDocumentCommand\citeproctext{}{}
\NewDocumentCommand\citeproc{mm}{%
  \begingroup\def\citeproctext{#2}\cite{#1}\endgroup}
\makeatletter
 % allow citations to break across lines
 \let\@cite@ofmt\@firstofone
 % avoid brackets around text for \cite:
 \def\@biblabel#1{}
 \def\@cite#1#2{{#1\if@tempswa , #2\fi}}
\makeatother
\newlength{\cslhangindent}
\setlength{\cslhangindent}{1.5em}
\newlength{\csllabelwidth}
\setlength{\csllabelwidth}{3em}
\newenvironment{CSLReferences}[2] % #1 hanging-indent, #2 entry-spacing
 {\begin{list}{}{%
  \setlength{\itemindent}{0pt}
  \setlength{\leftmargin}{0pt}
  \setlength{\parsep}{0pt}
  % turn on hanging indent if param 1 is 1
  \ifodd #1
   \setlength{\leftmargin}{\cslhangindent}
   \setlength{\itemindent}{-1\cslhangindent}
  \fi
  % set entry spacing
  \setlength{\itemsep}{#2\baselineskip}}}
 {\end{list}}
\usepackage{calc}
\newcommand{\CSLBlock}[1]{\hfill\break\parbox[t]{\linewidth}{\strut\ignorespaces#1\strut}}
\newcommand{\CSLLeftMargin}[1]{\parbox[t]{\csllabelwidth}{\strut#1\strut}}
\newcommand{\CSLRightInline}[1]{\parbox[t]{\linewidth - \csllabelwidth}{\strut#1\strut}}
\newcommand{\CSLIndent}[1]{\hspace{\cslhangindent}#1}
\setlength{\emergencystretch}{3em} % prevent overfull lines
\providecommand{\tightlist}{%
  \setlength{\itemsep}{0pt}\setlength{\parskip}{0pt}}
\usepackage{bookmark}
\IfFileExists{xurl.sty}{\usepackage{xurl}}{} % add URL line breaks if available
\urlstyle{same}
\hypersetup{
  pdftitle={AMRScan: A hybrid R and Nextflow toolkit for rapid antimicrobial resistance gene detection from sequencing data},
  pdfauthor={Kaitao Lai1},
  hidelinks,
  pdfcreator={LaTeX via pandoc}}

\title{AMRScan: A hybrid R and Nextflow toolkit for rapid antimicrobial
resistance gene detection from sequencing data}
\author{Kaitao Lai\textsuperscript{1}}
\date{2025-07-08}

\begin{document}
\maketitle

\textsuperscript{1} University of Sydney

\section{Summary}\label{summary}

\textbf{AMRScan} is a hybrid bioinformatics toolkit implemented in both
R and \href{https://www.nextflow.io/}{Nextflow} for the rapid and
reproducible detection of antimicrobial resistance (AMR) genes from
next-generation sequencing (NGS) data. The toolkit enables users to
identify AMR gene hits in sequencing reads by aligning them against
reference databases such as CARD using BLAST (Altschul et al. 1990).

The R implementation provides a concise, script-based approach suitable
for single-sample analysis, teaching, and rapid prototyping. In
contrast, the Nextflow implementation enables reproducible, scalable
workflows for multi-sample batch processing in high-performance
computing (HPC) and containerized environments. It leverages modular
pipeline design with support for automated database setup, quality
control, conversion, BLAST alignment, and results parsing.

AMRScan helps bridge the gap between lightweight exploratory analysis
and production-ready surveillance pipelines, making it suitable for both
research and public health genomics applications.

\section{Statement of Need}\label{statement-of-need}

While several large-scale AMR detection platforms such as ResFinder
(Zankari et al. 2012) exist, many are resource-intensive or require
complex installations. AMRScan addresses the need for a minimal,
transparent, and reproducible toolkit that can be used flexibly in small
labs, clinical settings, or large-scale surveillance workflows.

The inclusion of a pure Nextflow implementation enables high-throughput,
multi-sample analyses in cloud and HPC environments, while the
standalone R script remains accessible to users in data science,
microbiology, and epidemiology. Both versions use shared components
(e.g., a BLAST parsing script) to ensure consistency and reproducibility
of results.

\section{Usage Guidance}\label{usage-guidance}

AMRScan provides two usage modes, tailored to user needs:

\begin{itemize}
\item
  \textbf{R script (\texttt{AMRScan\_standalone.R})}: Best suited for
  small datasets, single-sample analysis, quick local tests, educational
  use, and lightweight environments without workflow managers.
\item
  \textbf{Nextflow workflow (\texttt{main.nf})}: Designed for
  large-scale, automated analyses, this version excels in multi-sample
  settings, HPC/cloud infrastructure, and environments where
  reproducibility, parallelism, and containerization are priorities.
\end{itemize}

This flexible dual-mode implementation ensures that AMRScan can serve
both teaching/demo scenarios and production-grade bioinformatics
pipelines.

\section{Implementation}\label{implementation}

\begin{itemize}
\tightlist
\item
  The R script \texttt{scripts/AMRScan\_standalone.R} encapsulates the
  entire pipeline in a linear script-based format.
\item
  The Nextflow workflow \texttt{workflow/main.nf} organizes the same
  logic into modular processes:

  \begin{itemize}
  \tightlist
  \item
    \texttt{DownloadCARD}, \texttt{MakeBLASTdb}, \texttt{ConvertFASTQ},
    \texttt{RunBLAST}, and \texttt{ParseResults}
  \end{itemize}
\item
  The shared R script \texttt{scripts/parse\_blast.R} is used for
  post-BLAST result summarization.
\item
  Both implementations are documented, testable, and validated using
  mock NGS input.
\end{itemize}

\section{Example Dataset and
Demonstration}\label{example-dataset-and-demonstration}

The example data used to validate AMRScan was obtained from a study by
Munim et al.~(2024) on multidrug-resistant \emph{Klebsiella pneumoniae}
isolated from poultry in Noakhali, Bangladesh (Munim et al. 2024). The
assembled genome was downloaded from GenBank
(\href{https://www.ncbi.nlm.nih.gov/assembly/GCA_037966445.1}{GCA\_037966445.1}).

For antimicrobial resistance gene detection, we used the protein homolog
model from the Comprehensive Antibiotic Resistance Database (CARD) (Jia
et al. 2017), version Broadstreet v4.0.1, available at:\\
\url{https://card.mcmaster.ca/download/0/broadstreet-v4.0.1.tar.bz2}

A sample output summary is shown below:

\subsection{Top AMR Hits Summary}\label{top-amr-hits-summary}

\begingroup
\small  % can also try \footnotesize, \scriptsize, or \tiny

\begin{longtable}[]{@{}
  >{\raggedright\arraybackslash}p{(\linewidth - 12\tabcolsep) * \real{0.1870}}
  >{\raggedright\arraybackslash}p{(\linewidth - 12\tabcolsep) * \real{0.0732}}
  >{\raggedright\arraybackslash}p{(\linewidth - 12\tabcolsep) * \real{0.0813}}
  >{\raggedright\arraybackslash}p{(\linewidth - 12\tabcolsep) * \real{0.0650}}
  >{\raggedright\arraybackslash}p{(\linewidth - 12\tabcolsep) * \real{0.0650}}
  >{\raggedright\arraybackslash}p{(\linewidth - 12\tabcolsep) * \real{0.0813}}
  >{\raggedright\arraybackslash}p{(\linewidth - 12\tabcolsep) * \real{0.4472}}@{}}
\toprule\noalign{}
\begin{minipage}[b]{\linewidth}\raggedright
Query
\end{minipage} & \begin{minipage}[b]{\linewidth}\raggedright
Subject
\end{minipage} & \begin{minipage}[b]{\linewidth}\raggedright
Identity
\end{minipage} & \begin{minipage}[b]{\linewidth}\raggedright
Length
\end{minipage} & \begin{minipage}[b]{\linewidth}\raggedright
Evalue
\end{minipage} & \begin{minipage}[b]{\linewidth}\raggedright
Bitscore
\end{minipage} & \begin{minipage}[b]{\linewidth}\raggedright
Annotation
\end{minipage} \\
\midrule\noalign{}
\endhead
\bottomrule\noalign{}
\endlastfoot
JBBPBW010000028.1 & OprA & 40.839 & 453 & 0 & 252 & OprA
\(Pseudomonas aeruginosa\) \\
JBBPBW010000035.1 & LAP-2 & 100.000 & 285 & 0 & 587 & LAP-2
\(Enterobacter cloacae\) \\
JBBPBW010000001.1 & SHV-11 & 100.000 & 286 & 0 & 581 & SHV-11
\(Klebsiella pneumoniae\) \\
JBBPBW010000010.1 & eptB & 99.303 & 574 & 0 & 1109 & eptB
\(Klebsiella pneumoniae subsp. rhinoscleromatis\) \\
JBBPBW010000104.1 & dfrA14 & 98.726 & 157 & 0 & 327 & dfrA14
\(Escherichia coli\) \\
JBBPBW010000011.1 & YojI & 83.912 & 547 & 0 & 885 & YojI
\(Escherichia coli str. K-12 substr. MG1655\) \\
\end{longtable}

\endgroup

\section{Software Repository}\label{software-repository}

The source code for AMRScan is freely available on GitHub at:\\
\url{https://github.com/biosciences/AMRScan}

\section{Acknowledgements}\label{acknowledgements}

The author thanks collaborators at University of Sydney for feedback on
early concepts.

\section*{References}\label{references}
\addcontentsline{toc}{section}{References}

\protect\phantomsection\label{refs}
\begin{CSLReferences}{1}{0}
\bibitem[\citeproctext]{ref-altschul1990blast}
Altschul, Stephen F, Warren Gish, Webb Miller, Eugene W Myers, and David
J Lipman. 1990. {``Basic Local Alignment Search Tool.''} \emph{Journal
of Molecular Biology} 215 (3): 403--10.
\url{https://doi.org/10.1016/S0022-2836(05)80360-2}.

\bibitem[\citeproctext]{ref-jia2017card}
Jia, Baofeng, Aliyu R Raphenya, Brian Alcock, Nicholas Waglechner, Peng
Guo, Katelyn K Tsang, Brendan A Lago, et al. 2017. {``CARD 2017:
Expansion and Model-Centric Curation of the Comprehensive Antibiotic
Resistance Database.''} \emph{Nucleic Acids Research} 45 (D1): D566--73.
\url{https://doi.org/10.1093/nar/gkw1004}.

\bibitem[\citeproctext]{ref-munim2024mdr}
Munim, Md Adnan, Afroza Akter Tanni, Md Mobarok Hossain, Kallyan Chakma,
Adnan Mannan, SM Rafiqul Islam, Jully Gogoi Tiwari, and Shipan Das
Gupta. 2024. {``Whole Genome Sequencing of Multidrug-Resistant
Klebsiella Pneumoniae from Poultry in Noakhali, Bangladesh: Assessing
Risk of Transmission to Humans in a Pilot Study.''} \emph{Comparative
Immunology, Microbiology and Infectious Diseases} 114: 102246.
\url{https://doi.org/10.1016/j.cimid.2024.102246}.

\bibitem[\citeproctext]{ref-zankari2012resfinder}
Zankari, Ea, Henrik Hasman, Salvatore Cosentino, Martin Vestergaard,
Simon Rasmussen, Ole Lund, Frank M Aarestrup, and Mette V Larsen. 2012.
{``Identification of Acquired Antimicrobial Resistance Genes.''}
\emph{Journal of Antimicrobial Chemotherapy} 67 (11): 2640--44.
\url{https://doi.org/10.1093/jac/dks261}.

\end{CSLReferences}

\end{document}
